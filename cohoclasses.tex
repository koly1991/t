\documentclass[paper=a4, fontsize=11pt]{scrartcl} 
\usepackage[T1]{fontenc} 
\usepackage{tikz-cd}
\usepackage{fourier} 
\usepackage[english]{babel} 
\usepackage{amsmath,amsfonts,amsthm} 
\usepackage{lipsum} 
\usepackage{sectsty} 
\allsectionsfont{\centering \normalfont\scshape} 
\usepackage{fancyhdr} 
\pagestyle{fancyplain} 
\fancyhead{} 
\fancyfoot[L]{} 
\fancyfoot[C]{} 
\fancyfoot[R]{\thepage} 
\renewcommand{\headrulewidth}{0pt} 
\renewcommand{\footrulewidth}{0pt} 
\setlength{\headheight}{13.6pt} 
\DeclareMathOperator{\id}{\text{id}}
\DeclareMathOperator{\Gal}{\text{Gal}}
\DeclareMathOperator{\Tr}{\text{Tr}}
\DeclareMathOperator{\Z}{\mathbb{Z}}
\DeclareMathOperator{\mmm}{\text{ mod }}
\DeclareMathOperator{\hoho}{\text{H}}
\numberwithin{equation}{section} 
\numberwithin{figure}{section} 
\numberwithin{table}{section} 
\setlength\parindent{0pt} 
\newcommand{\horrule}[1]{\rule{\linewidth}{#1}} 
\begin{document}
\section{Kolyvagin's construction of cohomology classes}
The congruences
$$a_\ell\equiv\ell+1\equiv 0\mmm p$$
follows from the fact that $\ell$ is inert in $K$.\par %misteri
Let $G_{K_1}=\Gal(K_f/K_1)$. Recall that for a prime $\ell$ with $\ell g=f$ the subgroup $G_{K_g}=\Gal(K_f/K_g)\subset G_{K_1}$ is a cyclic of order $\ell+1$. Now we define some elements in the group ring $\Z[G]$.\par The \textit{augmentation ideal} of the group ring Z$[G_\ell]$ is the kernel of the \textit{augmentation map}
\begin{center}
\begin{tikzcd}
	\epsilon: \mathbb{Z}[G_\ell] \ar[r,mapsto] & \mathbb{Z} \\[-2em] 
	\sum_{\sigma\in{G_\ell}} n_\sigma\sigma\ar[r,mapsto]  & \sum_{\sigma}n_\sigma \\[-2em]
\end{tikzcd}
\end{center}
Note that the kernel $\ker \epsilon$ is the $\mathbb{Z}$-submodule generated by elements of the form $\sigma-\sigma'$ for $\sigma,\sigma'\in G_\ell$. Now fix a generator $\sigma_\ell\in G_\ell$ and write
$$\sigma-\sigma'=\sigma_\ell^t-\sigma_\ell^s=\sum_{i=0}^{t-s-1}\sigma_\ell^{t-i}-\sigma_\ell^{t-i-1}=(\sigma_\ell-\text{id})\sigma''$$
for any $\sigma,\sigma'$ and some $\sigma''$. It follows that $\ker\epsilon=\mathbb{Z}[G_\ell]\cdot (\sigma_\ell-\text{id})$ is principal. Let $\text{Tr}_\ell=\sum_{\sigma\in G_\ell}\sigma\in \mathbb{Z}[G_\ell]$ and let $S_\ell$ be a solution of the following equation in $\mathbb{Z}[G_\ell]$
\begin{equation}\label{equation.kolyvagin}
	(\sigma_\ell-\text{id})\cdot S_\ell=(\ell+1)\text{id}-\text{Tr}_\ell
\end{equation}
Write $\tau_j=\text{id}+\sigma_\ell+\sigma_\ell^2+\cdots+\sigma_\ell^{j-1}$ and note that $(\sigma_\ell-\text{id})\cdot \tau_j=\sigma_\ell^j-\text{id}$. It follows that
$$(\sigma_\ell-\text{id})\sum_{j=1}^{\ell+1} \tau_j=\sum_{j=1}^{\ell+1} \sigma_\ell^j-(\ell+1)=\text{Tr}_\ell-(\ell+1)\text{id}$$
thus $S_\ell=-\sum_{j=1}^{\ell+1} \tau_j$ is a solution of equation \ref{equation.kolyvagin}. 
Note that the class of $S_\ell$ in $\Z[G_\ell]/\Z\cdot \Tr_\ell$ is well-defined since $(\sigma_\ell-\id)\cdot \Tr_\ell=0$. Let
$$S_f=\prod_{\ell\mid f}S_\ell$$
Note that $S_f$ is well-defined since the $S_\ell$ commute. Recall the construction of $y_f$ in section \ref{section.heegnerpoints}. Then the class of the point $S_f y_f\in E(K_f)$ in $E(K_f)/pE(K_f)$ is invariant by $G$, since for $\ell\mid f$ one has
%misteri suffices for \sigma_\ell
$$(\sigma_\ell-\id)S_fy_f=(\sigma_\ell-\id)S_\ell S_g y_f=\Big((\ell+1)\id-\Tr_\ell\Big)S_g y_f\equiv (0 - \Tr_\ell)S_gy_f\mmm pE(K_f)$$
and it suffices to check invariance for $\sigma_\ell$ and $\ell+1\equiv 0\mmm p $. But $\Tr_\ell S_g=S_g\Tr_\ell$ so
$$(\sigma_\ell-\id)S_fy_f\equiv - S_g\Tr_\ell y_f\mmm pE(K_f)$$
But $a_\ell\equiv 0 \mmm p$ and
$$\Tr_\ell y_f=a_\ell y_g\in pE(K_g)\subset pE(K_f)$$ 
It follows that the class $[S_f y_f]\in E(K_f)/pE(K_f)$ is well-defined. Moreover, the point $P_f=\sum_{\sigma\in G_K/G_{K_1}}\sigma S_f y_f$ has a class $[P_f]\in E(K_f)/pE(K_f)$ invariant under the action of $G_K=\Gal(K_f/K)$.
$E$ has no $p$-torsion rational over $K_f$. %misteri
There is an isomorphism 
$$\hoho^1(K,E_p)\stackrel{\simeq}{\rightarrow}\hoho^1(K_f,E_p)^{G_K}$$ %misteri
The \textit{inflation restriction sequence}
% misteri la traca de algo es algo
\end{document}
%Note that for any two solutions $S_\ell$ and $S_\ell'$ of \ref{equation.kolyvagin} one has $S_\ell-S_\ell'\in \mathbb{Z}[\text{Tr}_\ell]$ 
