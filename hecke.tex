\documentclass[paper=a4, fontsize=11pt]{scrartcl} 
\usepackage[T1]{fontenc} 
\usepackage{tikz-cd}
\usepackage{fourier} 
\usepackage[english]{babel} 
\usepackage{amsmath,amsfonts,amsthm} 
\usepackage{lipsum} 
\usepackage{sectsty} 
\allsectionsfont{\centering \normalfont\scshape} 
\usepackage{fancyhdr} 
\pagestyle{fancyplain} 
\fancyhead{} 
\fancyfoot[L]{} 
\fancyfoot[C]{} 
\fancyfoot[R]{\thepage} 
\renewcommand{\headrulewidth}{0pt} 
\renewcommand{\footrulewidth}{0pt} 
\setlength{\headheight}{13.6pt} 
\DeclareMathOperator{\id}{\text{id}}
\DeclareMathOperator{\Gal}{\text{Gal}}
\DeclareMathOperator{\Tr}{\text{Tr}}
\DeclareMathOperator{\Z}{\mathbb{Z}}
\DeclareMathOperator{\mmm}{\text{ mod }}
\DeclareMathOperator{\Lat}{\text{Lat}}
\DeclareMathOperator{\C}{\mathbb{C}}
\DeclareMathOperator{\LLL}{\mathcal{L}}
\numberwithin{equation}{section} 
\numberwithin{figure}{section} 
\numberwithin{table}{section} 
\setlength\parindent{0pt} 
\newcommand{\horrule}[1]{\rule{\linewidth}{#1}} 
\begin{document}
\section{Hecke operators and Eichler-Shimura}
\subsection{Hecke operators}
Recall that $\Lat_{\C}$ is the set of lattices $\Lambda\subset \C$. Now let $$\LLL=\bigoplus_{\Lambda\in\Lat_{\C}}\Z\cdot\Lambda$$ be the free abelian group generated by $\Lambda\in\Lat_{\C}$. The \textit{Hecke operators} are a family $\{T(n)\}_{n\geq 1}$ of $\Z$-linear operators on $\LLL$ defined by
\begin{center}
\begin{tikzcd}
T(n) : &[-2em] \LLL\ar[r] & \LLL \\[-2em]
 & \Lambda\ar[r,mapsto] & \sum_{\Lambda'\text{ with }(\Lambda:\Lambda')=n}\Lambda'
\end{tikzcd}
\end{center}
If we define $R(n)$ by $R(n)(\Lambda)=n\Lambda$ then
\begin{itemize}
\item For $m,n$ relatively prime one has $T(m)\circ T(n)=T(mn)$
\item For $p$ prime and $n\geq 1$ one has $T(p^n)\circ T(p)=T(p^{n+1})+pR(p)\circ T(p^{n-1})$
\end{itemize}
It follows that the $T(p)$ together with $R(p)$ generate all the other Hecke operators.
\end{document}
